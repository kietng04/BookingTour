\documentclass[12pt,a4paper]{article}

% ===== PACKAGES =====
\usepackage[utf8]{vietnam}
\usepackage{graphicx}
\usepackage{listings}
\usepackage{xcolor}
\usepackage{hyperref}
\usepackage{geometry}
\usepackage{fancyhdr}
\usepackage{titlesec}
\usepackage{caption}
\usepackage{subcaption}
\usepackage{booktabs}
\usepackage{enumitem}
\usepackage{float}
\usepackage{longtable}
\usepackage{array}

% ===== PAGE SETUP =====
\geometry{
    left=3cm,
    right=2cm,
    top=2cm,
    bottom=2cm
}

% ===== IMAGE PATH =====
\graphicspath{{images/}}

% ===== HYPERREF SETUP =====
\hypersetup{
    colorlinks=true,
    linkcolor=blue,
    filecolor=magenta,
    urlcolor=cyan,
    pdftitle={Báo cáo BookingTour},
    pdfauthor={Nhóm 10},
}

% ===== CODE LISTING SETUP =====
\lstset{
    basicstyle=\ttfamily\small,
    backgroundcolor=\color{gray!10},
    frame=single,
    numbers=left,
    numberstyle=\tiny\color{gray},
    keywordstyle=\color{blue},
    commentstyle=\color{green!60!black},
    stringstyle=\color{orange},
    breaklines=true,
    showstringspaces=false,
    tabsize=2
}

% ===== HEADER & FOOTER =====
\pagestyle{fancy}
\fancyhf{}
\fancyhead[L]{Báo cáo đồ án J2EE}
\fancyhead[R]{Nhóm 10}
\fancyfoot[C]{\thepage}

% ===== DOCUMENT START =====
\begin{document}

% ===== TITLE PAGE =====
\begin{titlepage}
    \centering
    \vspace*{1cm}

    {\Large\textbf{TRƯỜNG ĐẠI HỌC SÀI GÒN}}\\[0.5cm]
    {\Large\textbf{KHOA CÔNG NGHỆ THÔNG TIN}}\\[1.5cm]

    % TODO: Insert university logo here
    % \includegraphics[width=0.25\textwidth]{images/others/logo.png}\\[1.5cm]

    {\huge\textbf{BÁO CÁO ĐỒ ÁN}}\\[0.5cm]
    {\Large\textbf{HỌC PHẦN: CHUYÊN ĐỀ J2EE}}\\[1.5cm]

    {\LARGE\textbf{ĐỀ TÀI: XÂY DỰNG HỆ THỐNG}}\\[0.3cm]
    {\LARGE\textbf{BOOKING TOUR DU LỊCH}}\\[2cm]

    \begin{flushleft}
    \textbf{GVHD:} ThS. Nguyễn Thanh Phước\\[1cm]

    \textbf{Nhóm sinh viên thực hiện:}\\[0.3cm]
    \begin{enumerate}
        \item 3122410001 - Diệp Thụy An
        \item 3122410193 - Nguyễn Phan Tuấn Kiệt
        \item 3122410200 - Phạm Văn Kiệt
        \item 3122560000 - Nguyễn Thanh Thảo
    \end{enumerate}
    \end{flushleft}

    \vfill
    {\large TP. HỒ CHÍ MINH, THÁNG 12/2024}
\end{titlepage}

% ===== LỜI CAM ĐOAN =====
\newpage
\section*{LỜI CAM ĐOAN}
\addcontentsline{toc}{section}{LỜI CAM ĐOAN}

Nhóm em xin cam đoan nội dung trong Đồ án học phần Chuyên Đề J2EE về đề tài \textbf{``Xây dựng hệ thống Booking Tour Du lịch''} là sản phẩm của nhóm. Những vấn đề được trình bày trong báo cáo là quá trình học tập, nghiên cứu, làm việc của nhóm. Tất cả các tài liệu tham khảo đều có xuất xứ rõ ràng và được trích dẫn hợp pháp.

Nhóm em xin chịu hoàn toàn trách nhiệm cho lời cam đoan của mình.

\vspace{1cm}
\textit{TP. Hồ Chí Minh, ngày 14 tháng 12 năm 2024}

\vspace{1cm}
\textbf{Người cam đoan}

\vspace{0.5cm}
\textbf{Diệp Thụy An}

\textbf{Nguyễn Phan Tuấn Kiệt}

\textbf{Phạm Văn Kiệt}

\textbf{Nguyễn Thanh Thảo}

% ===== TABLE OF CONTENTS =====
\newpage
\tableofcontents

% ===== LIST OF FIGURES =====
\newpage
\listoffigures

% ===== CHAPTER 1: CHỨC NĂNG HỆ THỐNG =====
\newpage
\section{CHỨC NĂNG HỆ THỐNG}

\subsection{Chức năng cơ bản}

\subsubsection{Đối với khách hàng}

\begin{itemize}
    \item Xem trang chủ và các thông tin về tour du lịch.
    \item Đăng ký / Đăng nhập vào hệ thống bằng nhiều tuỳ chọn:
    \begin{itemize}
        \item Tài khoản hệ thống
        \item Tài khoản GitHub
        \item Tài khoản Google
    \end{itemize}
    \item Đăng xuất khỏi hệ thống.
    \item Tìm kiếm và xem danh sách tour (lịch trình, giá tour, thông tin chi tiết,...).
    \item Xem chi tiết tour, lịch trình từng ngày.
    \item Đặt tour với thông tin hành khách đầy đủ.
    \item Thanh toán sau khi đặt tour (MoMo).
    \item Xem, tìm kiếm booking đã đặt.
    \item Xem lịch sử booking của bản thân trong hệ thống.
    \item Đánh giá và review tour đã tham gia.
    \item Xem review của người dùng khác về tour.
    \item Yêu cầu tour tùy chỉnh (custom tour request).
\end{itemize}

\subsubsection{Đối với người quản trị}

\begin{itemize}
    \item Đăng nhập với tài khoản được cấp quyền quản trị.
    \item Đăng xuất.
    \item Quản lý tour du lịch:
    \begin{itemize}
        \item Tạo, sửa, xóa tour
        \item Quản lý lịch trình từng ngày (schedules)
        \item Quản lý hình ảnh tour
        \item Quản lý khuyến mãi (discounts)
    \end{itemize}
    \item Quản lý lịch khởi hành (departures):
    \begin{itemize}
        \item Tạo ngày khởi hành mới
        \item Quản lý số chỗ còn lại
        \item Cập nhật trạng thái departure
    \end{itemize}
    \item Quản lý khách hàng và người dùng.
    \item Quản lý booking:
    \begin{itemize}
        \item Xem danh sách booking
        \item Xem thông tin hành khách
        \item Quản lý trạng thái booking
    \end{itemize}
    \item Quản lý review và đánh giá:
    \begin{itemize}
        \item Duyệt review (approve/reject)
        \item Xóa review không phù hợp
        \item Xem thống kê đánh giá
    \end{itemize}
    \item Quản lý yêu cầu custom tour.
    \item Báo cáo, thống kê:
    \begin{itemize}
        \item Thống kê doanh thu theo thời gian
        \item Thống kê booking theo trạng thái
        \item Top tours theo số lượng đặt
        \item Tỷ lệ lấp đầy departure
    \end{itemize}
    \item Export dữ liệu (PDF, Excel).
\end{itemize}

\subsection{Chức năng nâng cao}

\begin{itemize}
    \item Kiến trúc microservices với 6 services độc lập
    \item Service discovery với Eureka Server
    \item API Gateway để routing và load balancing
    \item Xác thực OAuth2 (GitHub, Google)
    \item JWT token-based authentication
    \item Async messaging với RabbitMQ
    \item Image upload và hosting với Cloudinary
    \item Email notification với RabbitMQ queue
    \item Review moderation workflow
    \item Dashboard với analytics và charts
    \item Containerization với Docker
\end{itemize}

% ===== CHAPTER 2: THIẾT KẾ HỆ THỐNG =====
\newpage
\section{THIẾT KẾ HỆ THỐNG}

\subsection{Kiến trúc tổng thể}

Hệ thống BookingTour được thiết kế theo kiến trúc \textbf{Microservices}, bao gồm 6 services chính và 2 ứng dụng frontend. Tất cả các services được đăng ký với Eureka Server và giao tiếp qua API Gateway.

\begin{figure}[H]
    \centering
    % TODO: Create architecture diagram
    % \includegraphics[width=0.9\textwidth]{images/diagrams/01_architecture.png}
    \fbox{\parbox{0.9\textwidth}{\centering\vspace{2cm}
    \textbf{[TODO: Architecture Diagram]}\\
    Sơ đồ kiến trúc microservices\\
    Bao gồm: Eureka, Gateway, 6 services, 2 frontends
    \vspace{2cm}}}
    \caption{Kiến trúc tổng thể hệ thống BookingTour}
    \label{fig:architecture}
\end{figure}

\subsection{Cơ sở dữ liệu}

Hệ thống sử dụng 3 database PostgreSQL riêng biệt để đảm bảo tính độc lập giữa các services:

\begin{itemize}
    \item \textbf{userdb/tourdb}: Shared database cho User Service và Tour Service
    \item \textbf{bookingdb}: Database cho Booking Service
    \item \textbf{paymentdb}: Database cho Payment Service
\end{itemize}

\begin{figure}[H]
    \centering
    % TODO: Extract database ERD from original document or create new one
    % \includegraphics[width=0.95\textwidth]{images/diagrams/02_database_erd.png}
    \fbox{\parbox{0.95\textwidth}{\centering\vspace{3cm}
    \textbf{[TODO: Database ERD Diagram]}\\
    Entity Relationship Diagram\\
    Các bảng chính: users, tours, departures, bookings, payments, reviews
    \vspace{3cm}}}
    \caption{Sơ đồ cơ sở dữ liệu}
    \label{fig:database-erd}
\end{figure}

\subsubsection{Các bảng chính}

\textbf{User Management (userdb):}
\begin{itemize}
    \item \texttt{users}: Thông tin người dùng
    \item \texttt{email\_verification}: Xác thực email
\end{itemize}

\textbf{Tour Management (tourdb):}
\begin{itemize}
    \item \texttt{tours}: Thông tin tour
    \item \texttt{tour\_schedules}: Lịch trình từng ngày
    \item \texttt{tour\_images}: Hình ảnh tour
    \item \texttt{departures}: Ngày khởi hành
    \item \texttt{tour\_discounts}: Khuyến mãi
    \item \texttt{tour\_reviews}: Đánh giá và review
    \item \texttt{regions}, \texttt{provinces}: Địa điểm
    \item \texttt{custom\_tours}: Yêu cầu tour tùy chỉnh
\end{itemize}

\textbf{Booking Management (bookingdb):}
\begin{itemize}
    \item \texttt{bookings}: Thông tin booking
    \item \texttt{booking\_guests}: Thông tin hành khách
\end{itemize}

\textbf{Payment Management (paymentdb):}
\begin{itemize}
    \item \texttt{payments}: Giao dịch thanh toán
\end{itemize}

\subsection{Công nghệ sử dụng}

\begin{figure}[H]
    \centering
    % TODO: Create technology stack diagram
    % \includegraphics[width=0.85\textwidth]{images/diagrams/03_tech_stack.png}
    \fbox{\parbox{0.85\textwidth}{\centering\vspace{2.5cm}
    \textbf{[TODO: Technology Stack Diagram]}\\
    Backend: Spring Boot, Spring Cloud, PostgreSQL, RabbitMQ\\
    Frontend: React, Vite, TailwindCSS\\
    Infrastructure: Docker, Eureka, Cloudinary
    \vspace{2.5cm}}}
    \caption{Tổng quan công nghệ được sử dụng}
    \label{fig:tech-stack}
\end{figure}

\subsubsection{Backend}

\begin{itemize}
    \item \textbf{Spring Boot 3.3.3}: Framework chính cho microservices
    \item \textbf{Spring Cloud 2023.0.3}: Service discovery, API Gateway
    \item \textbf{Spring Security + JWT}: Authentication và authorization
    \item \textbf{Spring Mail}: Email notification
    \item \textbf{Spring Data JPA}: ORM với PostgreSQL
    \item \textbf{RabbitMQ}: Message broker cho async messaging
    \item \textbf{Netflix Eureka}: Service registry và discovery
    \item \textbf{Cloudinary}: Image storage và CDN
\end{itemize}

\subsubsection{Database}

\begin{itemize}
    \item \textbf{PostgreSQL 15}: Relational database chính
\end{itemize}

\subsubsection{Frontend}

\textbf{Client Frontend (React):}
\begin{itemize}
    \item \textbf{React 18.2.0}: UI framework
    \item \textbf{Vite 5.0.10+}: Build tool
    \item \textbf{TailwindCSS 3.4.1}: Styling
    \item \textbf{react-router-dom 6.21.3}: Routing
    \item \textbf{react-hook-form 7.65.0}: Form handling
    \item \textbf{framer-motion 12.23.24}: Animations
    \item \textbf{lucide-react 0.424.0}: Icons
\end{itemize}

\textbf{Admin Frontend (React):}
\begin{itemize}
    \item \textbf{React 18.2.0}: UI framework
    \item \textbf{Vite}: Build tool
    \item \textbf{TailwindCSS}: Styling
    \item \textbf{Recharts 2.7.2}: Data visualization
    \item \textbf{Playwright 1.48.2}: E2E testing
\end{itemize}

\subsubsection{Infrastructure \& DevOps}

\begin{itemize}
    \item \textbf{Docker}: Containerization
    \item \textbf{Docker Compose}: Multi-container orchestration
    \item \textbf{Java 17}: Runtime environment
    \item \textbf{Maven}: Build automation
\end{itemize}

\subsection{Message Queue Architecture}

Hệ thống sử dụng RabbitMQ để xử lý các tác vụ bất đồng bộ:

\begin{itemize}
    \item \textbf{Booking Events}: Xử lý đặt tour và reserve seats
    \item \textbf{Payment Events}: Xử lý thanh toán và callback
    \item \textbf{Email Events}: Gửi email xác nhận và thông báo
\end{itemize}

\begin{figure}[H]
    \centering
    % TODO: Create RabbitMQ flow diagram
    % \includegraphics[width=0.9\textwidth]{images/diagrams/04_rabbitmq_flow.png}
    \fbox{\parbox{0.9\textwidth}{\centering\vspace{2.5cm}
    \textbf{[TODO: RabbitMQ Event Flow Diagram]}\\
    Booking → Reserve Seats → Payment → Email\\
    Exchanges, Queues, Routing Keys
    \vspace{2.5cm}}}
    \caption{Luồng xử lý message với RabbitMQ}
    \label{fig:rabbitmq-flow}
\end{figure}

% ===== CHAPTER 3: CẤU TRÚC DỰ ÁN =====
\newpage
\section{CẤU TRÚC DỰ ÁN}

\subsection{Backend - Microservices}

\subsubsection{Eureka Server (Port 8761)}

Service discovery và registry cho tất cả các microservices.

\begin{figure}[H]
    \centering
    % TODO: Screenshot Eureka dashboard
    % \includegraphics[width=0.85\textwidth]{images/ui/01_eureka_dashboard.png}
    \fbox{\parbox{0.85\textwidth}{\centering\vspace{2cm}
    \textbf{[TODO: Eureka Dashboard Screenshot]}\\
    http://localhost:8761\\
    Hiển thị danh sách services đã đăng ký
    \vspace{2cm}}}
    \caption{Eureka Server Dashboard}
    \label{fig:eureka-dashboard}
\end{figure}

\textbf{Cấu trúc thư mục:}

\begin{itemize}
    \item \texttt{eureka-server/}
    \begin{itemize}
        \item \texttt{src/main/java/com/bookingtour/eureka/}
        \begin{itemize}
            \item \texttt{EurekaServerApplication.java}
        \end{itemize}
        \item \texttt{src/main/resources/}
        \begin{itemize}
            \item \texttt{application.yml}
        \end{itemize}
    \end{itemize}
\end{itemize}

\subsubsection{API Gateway (Port 8080)}

Single entry point cho tất cả các requests, routing và CORS configuration.

\begin{figure}[H]
    \centering
    % TODO: Code screenshot of application.yml routes
    % \includegraphics[width=0.9\textwidth]{images/code/01_gateway_routes.png}
    \fbox{\parbox{0.9\textwidth}{\centering\vspace{2cm}
    \textbf{[TODO: Gateway Routes Configuration]}\\
    api-gateway/src/main/resources/application.yml\\
    Các routes: /api/users, /api/tours, /api/bookings, etc.
    \vspace{2cm}}}
    \caption{API Gateway Routes Configuration}
    \label{fig:gateway-routes}
\end{figure}

\textbf{Key Routes:}
\begin{itemize}
    \item \texttt{/api/users/**} → user-service:8081
    \item \texttt{/api/tours/**} → tour-service:8082
    \item \texttt{/api/bookings/**} → booking-service:8083
    \item \texttt{/api/payments/**} → payment-service:8084
\end{itemize}

\subsubsection{User Service (Port 8081)}

Quản lý người dùng, authentication, OAuth2 integration.

\begin{figure}[H]
    \centering
    % TODO: Code screenshot of User Service structure
    % \includegraphics[width=0.7\textwidth]{images/code/02_user_service_structure.png}
    \fbox{\parbox{0.7\textwidth}{\centering\vspace{2.5cm}
    \textbf{[TODO: User Service Structure]}\\
    user-service/src/main/java/com/bookingtour/user/\\
    Controllers, Services, Models, Repositories
    \vspace{2.5cm}}}
    \caption{Cấu trúc User Service}
    \label{fig:user-service-structure}
\end{figure}

\textbf{Key Components:}
\begin{itemize}
    \item \texttt{AuthController}: Login, register, OAuth callbacks
    \item \texttt{UserController}: User CRUD operations
    \item \texttt{GithubOAuthService}: GitHub authentication
    \item \texttt{GoogleOAuthService}: Google authentication
    \item \texttt{EmailService}: Email notifications
    \item \texttt{JwtTokenProvider}: JWT token generation
\end{itemize}

\begin{figure}[H]
    \centering
    % TODO: Code screenshot of AuthController
    % \includegraphics[width=0.9\textwidth]{images/code/03_auth_controller.png}
    \fbox{\parbox{0.9\textwidth}{\centering\vspace{2.5cm}
    \textbf{[TODO: AuthController.java Source Code]}\\
    Các endpoints: /register, /login, /oauth/github, /oauth/google\\
    Lines: authentication logic
    \vspace{2.5cm}}}
    \caption{AuthController implementation}
    \label{fig:auth-controller}
\end{figure}

\subsubsection{Tour Service (Port 8082)}

Quản lý tour, schedules, images, reviews, departures.

\begin{figure}[H]
    \centering
    % TODO: Code screenshot of Tour Service structure
    % \includegraphics[width=0.7\textwidth]{images/code/04_tour_service_structure.png}
    \fbox{\parbox{0.7\textwidth}{\centering\vspace{2.5cm}
    \textbf{[TODO: Tour Service Structure]}\\
    tour-service/src/main/java/com/bookingtour/tour/\\
    Controllers, Services, Models, Repositories
    \vspace{2.5cm}}}
    \caption{Cấu trúc Tour Service}
    \label{fig:tour-service-structure}
\end{figure}

\textbf{Key Components:}
\begin{itemize}
    \item \texttt{TourController}: Tour CRUD operations
    \item \texttt{ReviewController}: Review management và moderation
    \item \texttt{DepartureController}: Departure management
    \item \texttt{ScheduleController}: Itinerary management
    \item \texttt{ImageController}: Image upload với Cloudinary
    \item \texttt{DiscountController}: Promotion management
    \item \texttt{ReservationRequestListener}: RabbitMQ listener cho seat reservation
\end{itemize}

\begin{figure}[H]
    \centering
    % TODO: Code screenshot of ReviewController
    % \includegraphics[width=0.9\textwidth]{images/code/05_review_controller.png}
    \fbox{\parbox{0.9\textwidth}{\centering\vspace{2.5cm}
    \textbf{[TODO: ReviewController.java Source Code]}\\
    Review CRUD, Admin moderation endpoints\\
    Workflow: PENDING → APPROVED/REJECTED
    \vspace{2.5cm}}}
    \caption{ReviewController với moderation workflow}
    \label{fig:review-controller}
\end{figure}

\subsubsection{Booking Service (Port 8083)}

Quản lý booking, guests, dashboard analytics.

\begin{figure}[H]
    \centering
    % TODO: Code screenshot of Booking Service structure
    % \includegraphics[width=0.7\textwidth]{images/code/06_booking_service_structure.png}
    \fbox{\parbox{0.7\textwidth}{\centering\vspace{2.5cm}
    \textbf{[TODO: Booking Service Structure]}\\
    booking-service/src/main/java/com/bookingtour/booking/\\
    Controllers, Services, Models, Repositories
    \vspace{2.5cm}}}
    \caption{Cấu trúc Booking Service}
    \label{fig:booking-service-structure}
\end{figure}

\textbf{Key Components:}
\begin{itemize}
    \item \texttt{BookingController}: Booking CRUD operations
    \item \texttt{DashboardController}: Analytics và statistics
    \item \texttt{ExportController}: PDF/Excel export
    \item \texttt{BookingEventPublisher}: Publish booking events
    \item \texttt{PaymentEventListener}: Listen payment results
    \item \texttt{TourEventListener}: Listen seat reservation results
\end{itemize}

\subsubsection{Payment Service (Port 8084)}

Xử lý thanh toán với MoMo integration.

\begin{figure}[H]
    \centering
    % TODO: Code screenshot of Payment Service structure
    % \includegraphics[width=0.7\textwidth]{images/code/07_payment_service_structure.png}
    \fbox{\parbox{0.7\textwidth}{\centering\vspace{2.5cm}
    \textbf{[TODO: Payment Service Structure]}\\
    payment-service/src/main/java/com/bookingtour/payment/\\
    Controllers, Services, Gateway integration
    \vspace{2.5cm}}}
    \caption{Cấu trúc Payment Service}
    \label{fig:payment-service-structure}
\end{figure}

\textbf{Key Components:}
\begin{itemize}
    \item \texttt{PaymentController}: Payment operations
    \item \texttt{MoMoCallbackController}: Webhook handling
    \item \texttt{MoMoGateway}: MoMo API integration
    \item \texttt{PaymentCommandListener}: Listen payment charge requests
    \item \texttt{PaymentEventPublisher}: Publish payment results
\end{itemize}

\subsection{Frontend}

\subsubsection{Client Frontend (Port 3000)}

Ứng dụng React cho khách hàng.

\begin{figure}[H]
    \centering
    % TODO: Code screenshot of frontend structure
    % \includegraphics[width=0.65\textwidth]{images/code/08_frontend_structure.png}
    \fbox{\parbox{0.65\textwidth}{\centering\vspace{3cm}
    \textbf{[TODO: Frontend Structure]}\\
    frontend/src/\\
    pages/, components/, services/, context/
    \vspace{3cm}}}
    \caption{Cấu trúc Client Frontend}
    \label{fig:frontend-structure}
\end{figure}

\textbf{Key Directories:}
\begin{itemize}
    \item \texttt{pages/}: Các trang chính (Home, TourDetail, Booking, etc.)
    \item \texttt{components/}: Reusable components
    \item \texttt{services/}: API calls và data fetching
    \item \texttt{context/}: Global state management (AuthContext)
\end{itemize}

\subsubsection{Admin Frontend (Port 5174)}

Dashboard quản trị với analytics.

\begin{figure}[H]
    \centering
    % TODO: Code screenshot of admin frontend structure
    % \includegraphics[width=0.65\textwidth]{images/code/09_admin_frontend_structure.png}
    \fbox{\parbox{0.65\textwidth}{\centering\vspace{3cm}
    \textbf{[TODO: Admin Frontend Structure]}\\
    frontend-admin/src/\\
    pages/, components/, layouts/
    \vspace{3cm}}}
    \caption{Cấu trúc Admin Frontend}
    \label{fig:admin-frontend-structure}
\end{figure}

\textbf{Key Modules:}
\begin{itemize}
    \item \texttt{Dashboard/}: Analytics và charts
    \item \texttt{Tours/}: Tour management
    \item \texttt{Departures/}: Departure management
    \item \texttt{Bookings/}: Booking management
    \item \texttt{Reviews/}: Review moderation
    \item \texttt{Users/}: User management
\end{itemize}

% ===== CHAPTER 4: THỰC NGHIỆM VÀ KẾT QUẢ =====
\newpage
\section{THỰC NGHIỆM VÀ KẾT QUẢ}

\subsection{Giao diện khách hàng}

\subsubsection{Đăng nhập và đăng ký}

\begin{figure}[H]
    \centering
    % TODO: Playwright screenshot
    % \includegraphics[width=0.7\textwidth]{images/ui/02_login.png}
    \fbox{\parbox{0.7\textwidth}{\centering\vspace{2.5cm}
    \textbf{[TODO: Login Page Screenshot]}\\
    http://localhost:3000/login\\
    Local login + OAuth buttons (GitHub, Google)
    \vspace{2.5cm}}}
    \caption{Giao diện đăng nhập}
    \label{fig:ui-login}
\end{figure}

\textbf{Chức năng:}
\begin{itemize}
    \item Đăng nhập bằng email + password
    \item OAuth integration (GitHub, Google)
    \item Form validation
    \item Error handling
\end{itemize}

\begin{figure}[H]
    \centering
    % TODO: Playwright screenshot
    % \includegraphics[width=0.7\textwidth]{images/ui/03_register.png}
    \fbox{\parbox{0.7\textwidth}{\centering\vspace{2.5cm}
    \textbf{[TODO: Register Page Screenshot]}\\
    http://localhost:3000/register\\
    Registration form với email verification
    \vspace{2.5cm}}}
    \caption{Giao diện đăng ký}
    \label{fig:ui-register}
\end{figure}

\textbf{Chức năng:}
\begin{itemize}
    \item Đăng ký với thông tin cơ bản
    \item Email verification (6-digit code)
    \item Password strength validation
\end{itemize}

\subsubsection{Trang chủ và tìm kiếm tour}

\begin{figure}[H]
    \centering
    % TODO: Playwright screenshot
    % \includegraphics[width=0.95\textwidth]{images/ui/04_homepage.png}
    \fbox{\parbox{0.95\textwidth}{\centering\vspace{2.5cm}
    \textbf{[TODO: Homepage Screenshot]}\\
    http://localhost:3000/\\
    Hero section, featured tours, search bar
    \vspace{2.5cm}}}
    \caption{Giao diện trang chủ}
    \label{fig:ui-homepage}
\end{figure}

\textbf{Chức năng:}
\begin{itemize}
    \item Hero section với call-to-action
    \item Featured tours carousel
    \item Quick search form
    \item Tour highlights
\end{itemize}

\begin{figure}[H]
    \centering
    % TODO: Playwright screenshot
    % \includegraphics[width=0.95\textwidth]{images/ui/05_tour_explore.png}
    \fbox{\parbox{0.95\textwidth}{\centering\vspace{2.5cm}
    \textbf{[TODO: Tour Explore Page Screenshot]}\\
    http://localhost:3000/tours\\
    Tour listing với filters và pagination
    \vspace{2.5cm}}}
    \caption{Giao diện tìm kiếm tour}
    \label{fig:ui-tour-explore}
\end{figure}

\textbf{Chức năng:}
\begin{itemize}
    \item Filter theo region, province
    \item Search by keyword
    \item Pagination
    \item Tour cards với thông tin cơ bản
\end{itemize}

\subsubsection{Chi tiết tour và reviews}

\begin{figure}[H]
    \centering
    % TODO: Playwright screenshot
    % \includegraphics[width=0.95\textwidth]{images/ui/06_tour_detail.png}
    \fbox{\parbox{0.95\textwidth}{\centering\vspace{3cm}
    \textbf{[TODO: Tour Detail Page Screenshot]}\\
    http://localhost:3000/tours/[slug]\\
    Gallery, itinerary, reviews, booking button
    \vspace{3cm}}}
    \caption{Giao diện chi tiết tour}
    \label{fig:ui-tour-detail}
\end{figure}

\textbf{Chức năng:}
\begin{itemize}
    \item Image gallery với hero image
    \item Tour information (duration, departure point, price)
    \item Day-by-day itinerary
    \item Reviews section
    \item Departure dates selection
    \item Book now button
\end{itemize}

\begin{figure}[H]
    \centering
    % TODO: Playwright screenshot
    % \includegraphics[width=0.9\textwidth]{images/ui/07_reviews.png}
    \fbox{\parbox{0.9\textwidth}{\centering\vspace{2.5cm}
    \textbf{[TODO: Reviews Page Screenshot]}\\
    http://localhost:3000/reviews\\
    Review listing, filter by rating, review summary
    \vspace{2.5cm}}}
    \caption{Giao diện reviews}
    \label{fig:ui-reviews}
\end{figure}

\textbf{Chức năng:}
\begin{itemize}
    \item Hiển thị approved reviews
    \item Filter by rating (1-5 stars)
    \item Review summary statistics
    \item User avatar và guest name
\end{itemize}

\subsubsection{Booking flow}

\begin{figure}[H]
    \centering
    % TODO: Playwright screenshot
    % \includegraphics[width=0.95\textwidth]{images/ui/08_booking.png}
    \fbox{\parbox{0.95\textwidth}{\centering\vspace{3cm}
    \textbf{[TODO: Booking Page Screenshot]}\\
    http://localhost:3000/booking\\
    Guest info form, seat selection, payment method
    \vspace{3cm}}}
    \caption{Giao diện đặt tour}
    \label{fig:ui-booking}
\end{figure}

\textbf{Chức năng:}
\begin{itemize}
    \item Guest information form
    \item Number of seats selection
    \item Payment method selection (MoMo)
    \item Total price calculation
    \item Terms and conditions checkbox
\end{itemize}

\begin{figure}[H]
    \centering
    % TODO: Playwright screenshot
    % \includegraphics[width=0.85\textwidth]{images/ui/09_payment_success.png}
    \fbox{\parbox{0.85\textwidth}{\centering\vspace{2.5cm}
    \textbf{[TODO: Payment Success Screenshot]}\\
    http://localhost:3000/payment-status?status=success\\
    Booking confirmation, booking details
    \vspace{2.5cm}}}
    \caption{Giao diện thanh toán thành công}
    \label{fig:ui-payment-success}
\end{figure}

\textbf{Chức năng:}
\begin{itemize}
    \item Success message
    \item Booking details display
    \item Link to booking history
    \item Email notification sent
\end{itemize}

\subsubsection{Booking history và My Reviews}

\begin{figure}[H]
    \centering
    % TODO: Playwright screenshot
    % \includegraphics[width=0.95\textwidth]{images/ui/10_booking_history.png}
    \fbox{\parbox{0.95\textwidth}{\centering\vspace{2.5cm}
    \textbf{[TODO: Booking History Screenshot]}\\
    http://localhost:3000/booking-history\\
    List of user's bookings với status
    \vspace{2.5cm}}}
    \caption{Giao diện lịch sử booking}
    \label{fig:ui-booking-history}
\end{figure}

\textbf{Chức năng:}
\begin{itemize}
    \item Hiển thị tất cả bookings của user
    \item Booking status (PENDING, CONFIRMED, FAILED)
    \item Guest information
    \item Total amount
\end{itemize}

\begin{figure}[H]
    \centering
    % TODO: Playwright screenshot
    % \includegraphics[width=0.9\textwidth]{images/ui/11_my_reviews.png}
    \fbox{\parbox{0.9\textwidth}{\centering\vspace{2.5cm}
    \textbf{[TODO: My Reviews Screenshot]}\\
    http://localhost:3000/my-reviews\\
    User's reviews với edit/delete buttons
    \vspace{2.5cm}}}
    \caption{Giao diện My Reviews}
    \label{fig:ui-my-reviews}
\end{figure}

\textbf{Chức năng:}
\begin{itemize}
    \item Hiển thị reviews của user
    \item Edit review (status → PENDING)
    \item Delete review
    \item Review status display
\end{itemize}

\subsection{Giao diện quản trị}

\subsubsection{Dashboard và Analytics}

\begin{figure}[H]
    \centering
    % TODO: Playwright screenshot
    % \includegraphics[width=0.95\textwidth]{images/ui/12_admin_dashboard.png}
    \fbox{\parbox{0.95\textwidth}{\centering\vspace{3cm}
    \textbf{[TODO: Admin Dashboard Screenshot]}\\
    http://localhost:5174/dashboard\\
    Revenue charts, booking stats, top tours
    \vspace{3cm}}}
    \caption{Giao diện dashboard quản trị}
    \label{fig:ui-admin-dashboard}
\end{figure}

\textbf{Chức năng:}
\begin{itemize}
    \item Revenue statistics chart (by date range)
    \item Booking count by status
    \item Top 5 tours by bookings
    \item Departure occupancy rates
    \item Recent bookings timeline
    \item Key metrics cards
\end{itemize}

\subsubsection{Quản lý Tours}

\begin{figure}[H]
    \centering
    % TODO: Playwright screenshot
    % \includegraphics[width=0.95\textwidth]{images/ui/13_admin_tours.png}
    \fbox{\parbox{0.95\textwidth}{\centering\vspace{2.5cm}
    \textbf{[TODO: Admin Tours List Screenshot]}\\
    http://localhost:5174/tours\\
    Tour table với CRUD actions
    \vspace{2.5cm}}}
    \caption{Giao diện quản lý tours}
    \label{fig:ui-admin-tours}
\end{figure}

\textbf{Chức năng:}
\begin{itemize}
    \item Tour listing table
    \item Create new tour button
    \item Edit/Delete actions
    \item Status management
    \item Pagination
\end{itemize}

\begin{figure}[H]
    \centering
    % TODO: Playwright screenshot
    % \includegraphics[width=0.9\textwidth]{images/ui/14_admin_tour_create.png}
    \fbox{\parbox{0.9\textwidth}{\centering\vspace{3cm}
    \textbf{[TODO: Admin Tour Create Screenshot]}\\
    http://localhost:5174/tours/create\\
    Tour form, schedule editor, image upload
    \vspace{3cm}}}
    \caption{Giao diện tạo tour mới}
    \label{fig:ui-admin-tour-create}
\end{figure}

\textbf{Chức năng:}
\begin{itemize}
    \item Tour information form
    \item Region và province selection
    \item Price input (adult/child)
    \item Image upload (Cloudinary)
    \item Schedule/itinerary editor
    \item Discount management
\end{itemize}

\subsubsection{Quản lý Departures}

\begin{figure}[H]
    \centering
    % TODO: Playwright screenshot
    % \includegraphics[width=0.95\textwidth]{images/ui/15_admin_departures.png}
    \fbox{\parbox{0.95\textwidth}{\centering\vspace{2.5cm}
    \textbf{[TODO: Admin Departures Screenshot]}\\
    http://localhost:5174/departures\\
    Departure table với seat availability
    \vspace{2.5cm}}}
    \caption{Giao diện quản lý departures}
    \label{fig:ui-admin-departures}
\end{figure}

\textbf{Chức năng:}
\begin{itemize}
    \item Departure listing
    \item Seat availability tracking (total/remaining)
    \item Status management (CONCHO, SAPFULL, FULL)
    \item Create/Edit/Delete departures
\end{itemize}

\subsubsection{Quản lý Bookings}

\begin{figure}[H]
    \centering
    % TODO: Playwright screenshot
    % \includegraphics[width=0.95\textwidth]{images/ui/16_admin_bookings.png}
    \fbox{\parbox{0.95\textwidth}{\centering\vspace{2.5cm}
    \textbf{[TODO: Admin Bookings Screenshot]}\\
    http://localhost:5174/bookings\\
    Booking table với guest info và payment status
    \vspace{2.5cm}}}
    \caption{Giao diện quản lý bookings}
    \label{fig:ui-admin-bookings}
\end{figure}

\textbf{Chức năng:}
\begin{itemize}
    \item Booking listing với pagination
    \item Booking status display
    \item Guest information
    \item Total amount
    \item Payment method
    \item View booking details
\end{itemize}

\textbf{Known Issues:}
\begin{itemize}
    \item BUG-1: Tour name hiển thị ``Tour \#X'' thay vì tên thật
    \item BUG-2: Username hiển thị ``Guest \#X'' thay vì username thật
\end{itemize}

\subsubsection{Quản lý Reviews (Moderation)}

\begin{figure}[H]
    \centering
    % TODO: Playwright screenshot
    % \includegraphics[width=0.95\textwidth]{images/ui/17_admin_reviews.png}
    \fbox{\parbox{0.95\textwidth}{\centering\vspace{2.5cm}
    \textbf{[TODO: Admin Reviews Screenshot]}\\
    http://localhost:5174/reviews\\
    Review moderation table với approve/reject buttons
    \vspace{2.5cm}}}
    \caption{Giao diện quản lý reviews}
    \label{fig:ui-admin-reviews}
\end{figure}

\textbf{Chức năng:}
\begin{itemize}
    \item Review listing (all statuses)
    \item Filter by status (PENDING, APPROVED, REJECTED)
    \item Filter by rating, tour
    \item Approve/Reject buttons
    \item Delete review
    \item View review details
\end{itemize}

\subsubsection{Quản lý Users}

\begin{figure}[H]
    \centering
    % TODO: Playwright screenshot
    % \includegraphics[width=0.95\textwidth]{images/ui/18_admin_users.png}
    \fbox{\parbox{0.95\textwidth}{\centering\vspace{2.5cm}
    \textbf{[TODO: Admin Users Screenshot]}\\
    http://localhost:5174/users\\
    User table với create/edit/disable actions
    \vspace{2.5cm}}}
    \caption{Giao diện quản lý users}
    \label{fig:ui-admin-users}
\end{figure}

\textbf{Chức năng:}
\begin{itemize}
    \item User listing
    \item Create/Edit user
    \item Disable user account
    \item User status (ACTIVE/INACTIVE)
    \item OAuth provider info
\end{itemize}

% ===== CHAPTER 5: KẾT LUẬN =====
\newpage
\section{KẾT LUẬN}

\subsection{Đóng góp của thành viên}

\begin{table}[H]
\centering
\caption{Bảng đóng góp của các thành viên}
\label{tab:contributions}
\begin{tabular}{|c|l|l|c|}
\hline
\textbf{STT} & \textbf{Mã sinh viên} & \textbf{Họ và tên} & \textbf{Đánh giá (\%)} \\
\hline
1 & 3122410001 & Diệp Thụy An & 25\% \\
\hline
2 & 3122410193 & Nguyễn Phan Tuấn Kiệt & 25\% \\
\hline
3 & 3122410200 & Phạm Văn Kiệt & 25\% \\
\hline
4 & 3122560000 & Nguyễn Thanh Thảo & 25\% \\
\hline
\end{tabular}
\end{table}

\subsection{Tổng kết dự án}

Đồ án ``Xây dựng hệ thống Booking Tour Du lịch'' đã hoàn thành các mục tiêu đề ra:

\textbf{Những gì đã đạt được:}
\begin{itemize}
    \item Xây dựng thành công kiến trúc microservices với 6 services độc lập
    \item Triển khai đầy đủ chức năng cho khách hàng và quản trị viên
    \item Tích hợp OAuth2 (GitHub, Google) cho authentication
    \item Xây dựng hệ thống review và rating với workflow moderation
    \item Tích hợp thanh toán MoMo wallet
    \item Triển khai async messaging với RabbitMQ
    \item Xây dựng dashboard analytics với charts
    \item Containerization với Docker
    \item Image hosting với Cloudinary
    \item Email notification system
\end{itemize}

\textbf{Hạn chế hiện tại:}
\begin{itemize}
    \item Chưa áp dụng đầy đủ JWT filter cho tất cả protected endpoints
    \item Role-based authorization chưa hoàn chỉnh
    \item Hai bugs nhỏ trong admin booking table
    \item Chưa có distributed tracing (Sleuth/Zipkin)
    \item Chưa có circuit breaker pattern
    \item Test coverage còn hạn chế
\end{itemize}

\subsection{Hướng phát triển}

Để nâng cao chất lượng và đáp ứng tốt hơn nhu cầu thực tế, hệ thống cần tiếp tục phát triển:

\textbf{Ngắn hạn:}
\begin{itemize}
    \item Fix 2 bugs trong admin booking table
    \item Hoàn thiện JWT filter và authorization
    \item Tăng cường unit và integration tests
    \item Implement API documentation (Swagger/OpenAPI)
\end{itemize}

\textbf{Trung hạn:}
\begin{itemize}
    \item Implement circuit breaker pattern (Resilience4j)
    \item Thêm distributed tracing (Sleuth + Zipkin)
    \item Centralized logging (ELK stack)
    \item Implement refund flow
    \item Thêm payment gateway: VNPay
\end{itemize}

\textbf{Dài hạn:}
\begin{itemize}
    \item Phát triển mobile application (React Native/Flutter)
    \item Implement real-time notifications (WebSocket)
    \item Thêm recommendation system
    \item Implement caching strategy (Redis)
    \item Performance optimization và load testing
    \item Security hardening và penetration testing
\end{itemize}

\subsection{Kết luận}

Dự án ``Xây dựng hệ thống Booking Tour Du lịch'' đã thành công trong việc áp dụng kiến trúc microservices hiện đại để xây dựng một hệ thống hoàn chỉnh. Hệ thống không chỉ đáp ứng các yêu cầu chức năng cơ bản mà còn triển khai nhiều tính năng nâng cao như OAuth2, async messaging, review moderation, và analytics dashboard.

Qua quá trình thực hiện đồ án, nhóm đã có cơ hội áp dụng các kiến thức về:
\begin{itemize}
    \item Spring Boot và Spring Cloud ecosystem
    \item Microservices architecture patterns
    \item Message-driven architecture với RabbitMQ
    \item OAuth2 và JWT authentication
    \item React và modern frontend development
    \item Docker containerization
    \item RESTful API design
\end{itemize}

Nhóm cam kết sẽ tiếp tục phát triển và hoàn thiện hệ thống trong tương lai.

% ===== REFERENCES (Optional) =====
\newpage
\section*{TÀI LIỆU THAM KHẢO}
\addcontentsline{toc}{section}{TÀI LIỆU THAM KHẢO}

\begin{enumerate}
    \item Spring Boot Documentation, \url{https://spring.io/projects/spring-boot}
    \item Spring Cloud Documentation, \url{https://spring.io/projects/spring-cloud}
    \item React Documentation, \url{https://react.dev/}
    \item PostgreSQL Documentation, \url{https://www.postgresql.org/docs/}
    \item RabbitMQ Documentation, \url{https://www.rabbitmq.com/documentation.html}
    \item Docker Documentation, \url{https://docs.docker.com/}
    \item MoMo Payment API Documentation
    \item Cloudinary Documentation, \url{https://cloudinary.com/documentation}
\end{enumerate}

\end{document}
